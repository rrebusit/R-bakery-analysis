% Options for packages loaded elsewhere
\PassOptionsToPackage{unicode}{hyperref}
\PassOptionsToPackage{hyphens}{url}
%
\documentclass[
]{article}
\usepackage{amsmath,amssymb}
\usepackage{lmodern}
\usepackage{iftex}
\ifPDFTeX
  \usepackage[T1]{fontenc}
  \usepackage[utf8]{inputenc}
  \usepackage{textcomp} % provide euro and other symbols
\else % if luatex or xetex
  \usepackage{unicode-math}
  \defaultfontfeatures{Scale=MatchLowercase}
  \defaultfontfeatures[\rmfamily]{Ligatures=TeX,Scale=1}
\fi
% Use upquote if available, for straight quotes in verbatim environments
\IfFileExists{upquote.sty}{\usepackage{upquote}}{}
\IfFileExists{microtype.sty}{% use microtype if available
  \usepackage[]{microtype}
  \UseMicrotypeSet[protrusion]{basicmath} % disable protrusion for tt fonts
}{}
\makeatletter
\@ifundefined{KOMAClassName}{% if non-KOMA class
  \IfFileExists{parskip.sty}{%
    \usepackage{parskip}
  }{% else
    \setlength{\parindent}{0pt}
    \setlength{\parskip}{6pt plus 2pt minus 1pt}}
}{% if KOMA class
  \KOMAoptions{parskip=half}}
\makeatother
\usepackage{xcolor}
\usepackage[margin=1in]{geometry}
\usepackage{color}
\usepackage{fancyvrb}
\newcommand{\VerbBar}{|}
\newcommand{\VERB}{\Verb[commandchars=\\\{\}]}
\DefineVerbatimEnvironment{Highlighting}{Verbatim}{commandchars=\\\{\}}
% Add ',fontsize=\small' for more characters per line
\usepackage{framed}
\definecolor{shadecolor}{RGB}{248,248,248}
\newenvironment{Shaded}{\begin{snugshade}}{\end{snugshade}}
\newcommand{\AlertTok}[1]{\textcolor[rgb]{0.94,0.16,0.16}{#1}}
\newcommand{\AnnotationTok}[1]{\textcolor[rgb]{0.56,0.35,0.01}{\textbf{\textit{#1}}}}
\newcommand{\AttributeTok}[1]{\textcolor[rgb]{0.77,0.63,0.00}{#1}}
\newcommand{\BaseNTok}[1]{\textcolor[rgb]{0.00,0.00,0.81}{#1}}
\newcommand{\BuiltInTok}[1]{#1}
\newcommand{\CharTok}[1]{\textcolor[rgb]{0.31,0.60,0.02}{#1}}
\newcommand{\CommentTok}[1]{\textcolor[rgb]{0.56,0.35,0.01}{\textit{#1}}}
\newcommand{\CommentVarTok}[1]{\textcolor[rgb]{0.56,0.35,0.01}{\textbf{\textit{#1}}}}
\newcommand{\ConstantTok}[1]{\textcolor[rgb]{0.00,0.00,0.00}{#1}}
\newcommand{\ControlFlowTok}[1]{\textcolor[rgb]{0.13,0.29,0.53}{\textbf{#1}}}
\newcommand{\DataTypeTok}[1]{\textcolor[rgb]{0.13,0.29,0.53}{#1}}
\newcommand{\DecValTok}[1]{\textcolor[rgb]{0.00,0.00,0.81}{#1}}
\newcommand{\DocumentationTok}[1]{\textcolor[rgb]{0.56,0.35,0.01}{\textbf{\textit{#1}}}}
\newcommand{\ErrorTok}[1]{\textcolor[rgb]{0.64,0.00,0.00}{\textbf{#1}}}
\newcommand{\ExtensionTok}[1]{#1}
\newcommand{\FloatTok}[1]{\textcolor[rgb]{0.00,0.00,0.81}{#1}}
\newcommand{\FunctionTok}[1]{\textcolor[rgb]{0.00,0.00,0.00}{#1}}
\newcommand{\ImportTok}[1]{#1}
\newcommand{\InformationTok}[1]{\textcolor[rgb]{0.56,0.35,0.01}{\textbf{\textit{#1}}}}
\newcommand{\KeywordTok}[1]{\textcolor[rgb]{0.13,0.29,0.53}{\textbf{#1}}}
\newcommand{\NormalTok}[1]{#1}
\newcommand{\OperatorTok}[1]{\textcolor[rgb]{0.81,0.36,0.00}{\textbf{#1}}}
\newcommand{\OtherTok}[1]{\textcolor[rgb]{0.56,0.35,0.01}{#1}}
\newcommand{\PreprocessorTok}[1]{\textcolor[rgb]{0.56,0.35,0.01}{\textit{#1}}}
\newcommand{\RegionMarkerTok}[1]{#1}
\newcommand{\SpecialCharTok}[1]{\textcolor[rgb]{0.00,0.00,0.00}{#1}}
\newcommand{\SpecialStringTok}[1]{\textcolor[rgb]{0.31,0.60,0.02}{#1}}
\newcommand{\StringTok}[1]{\textcolor[rgb]{0.31,0.60,0.02}{#1}}
\newcommand{\VariableTok}[1]{\textcolor[rgb]{0.00,0.00,0.00}{#1}}
\newcommand{\VerbatimStringTok}[1]{\textcolor[rgb]{0.31,0.60,0.02}{#1}}
\newcommand{\WarningTok}[1]{\textcolor[rgb]{0.56,0.35,0.01}{\textbf{\textit{#1}}}}
\usepackage{graphicx}
\makeatletter
\def\maxwidth{\ifdim\Gin@nat@width>\linewidth\linewidth\else\Gin@nat@width\fi}
\def\maxheight{\ifdim\Gin@nat@height>\textheight\textheight\else\Gin@nat@height\fi}
\makeatother
% Scale images if necessary, so that they will not overflow the page
% margins by default, and it is still possible to overwrite the defaults
% using explicit options in \includegraphics[width, height, ...]{}
\setkeys{Gin}{width=\maxwidth,height=\maxheight,keepaspectratio}
% Set default figure placement to htbp
\makeatletter
\def\fps@figure{htbp}
\makeatother
\setlength{\emergencystretch}{3em} % prevent overfull lines
\providecommand{\tightlist}{%
  \setlength{\itemsep}{0pt}\setlength{\parskip}{0pt}}
\setcounter{secnumdepth}{-\maxdimen} % remove section numbering
\ifLuaTeX
  \usepackage{selnolig}  % disable illegal ligatures
\fi
\IfFileExists{bookmark.sty}{\usepackage{bookmark}}{\usepackage{hyperref}}
\IfFileExists{xurl.sty}{\usepackage{xurl}}{} % add URL line breaks if available
\urlstyle{same} % disable monospaced font for URLs
\hypersetup{
  pdftitle={Bakery Analysis},
  pdfauthor={Rica Rebusit},
  hidelinks,
  pdfcreator={LaTeX via pandoc}}

\title{Bakery Analysis}
\author{Rica Rebusit}
\date{2023-10-29}

\begin{document}
\maketitle

I will be using this dataset from Kaggle -\textgreater{}
\includegraphics{"https://www.kaggle.com/datasets/matthieugimbert/french-bakery-daily-sales/"}
to analyze and forecast sales of a french bakery from 2021 to 2022 based
on:

date: Date order time: Time order ticket number: Identifier for every
single transaction article: Name of the product sold quantity: quantity
sold unit\_price: price per product (In Euros)

There are 234,005 observations in the dataset as well.

\hypertarget{loading-necessary-libraries-and-importing-data}{%
\section{Loading Necessary Libraries and Importing
Data}\label{loading-necessary-libraries-and-importing-data}}

\begin{Shaded}
\begin{Highlighting}[]
\FunctionTok{library}\NormalTok{(tidyverse)}
\end{Highlighting}
\end{Shaded}

\begin{verbatim}
## -- Attaching packages --------------------------------------- tidyverse 1.3.1 --
\end{verbatim}

\begin{verbatim}
## v ggplot2 3.4.2     v purrr   0.3.4
## v tibble  3.1.6     v dplyr   1.0.8
## v tidyr   1.2.0     v stringr 1.4.0
## v readr   2.1.2     v forcats 0.5.1
\end{verbatim}

\begin{verbatim}
## Warning: package 'ggplot2' was built under R version 4.1.2
\end{verbatim}

\begin{verbatim}
## Warning: package 'tidyr' was built under R version 4.1.2
\end{verbatim}

\begin{verbatim}
## Warning: package 'readr' was built under R version 4.1.2
\end{verbatim}

\begin{verbatim}
## Warning: package 'dplyr' was built under R version 4.1.2
\end{verbatim}

\begin{verbatim}
## -- Conflicts ------------------------------------------ tidyverse_conflicts() --
## x dplyr::filter() masks stats::filter()
## x dplyr::lag()    masks stats::lag()
\end{verbatim}

\begin{Shaded}
\begin{Highlighting}[]
\NormalTok{bakery }\OtherTok{\textless{}{-}} \FunctionTok{read.csv}\NormalTok{(}\StringTok{"Bakery sales.csv"}\NormalTok{)}
\end{Highlighting}
\end{Shaded}

\hypertarget{eda-and-data-cleaning}{%
\section{EDA and Data Cleaning}\label{eda-and-data-cleaning}}

\begin{Shaded}
\begin{Highlighting}[]
\FunctionTok{head}\NormalTok{(bakery)}
\end{Highlighting}
\end{Shaded}

\begin{verbatim}
##    X       date  time ticket_number              article Quantity unit_price
## 1  0 2021-01-02 08:38        150040             BAGUETTE        1     0,90 €
## 2  1 2021-01-02 08:38        150040     PAIN AU CHOCOLAT        3     1,20 €
## 3  4 2021-01-02 09:14        150041     PAIN AU CHOCOLAT        2     1,20 €
## 4  5 2021-01-02 09:14        150041                 PAIN        1     1,15 €
## 5  8 2021-01-02 09:25        150042 TRADITIONAL BAGUETTE        5     1,20 €
## 6 11 2021-01-02 09:25        150043             BAGUETTE        2     0,90 €
\end{verbatim}

\begin{Shaded}
\begin{Highlighting}[]
\FunctionTok{str}\NormalTok{(bakery)}
\end{Highlighting}
\end{Shaded}

\begin{verbatim}
## 'data.frame':    234005 obs. of  7 variables:
##  $ X            : int  0 1 4 5 8 11 12 15 18 19 ...
##  $ date         : chr  "2021-01-02" "2021-01-02" "2021-01-02" "2021-01-02" ...
##  $ time         : chr  "08:38" "08:38" "09:14" "09:14" ...
##  $ ticket_number: num  150040 150040 150041 150041 150042 ...
##  $ article      : chr  "BAGUETTE" "PAIN AU CHOCOLAT" "PAIN AU CHOCOLAT" "PAIN" ...
##  $ Quantity     : num  1 3 2 1 5 2 3 1 3 6 ...
##  $ unit_price   : chr  "0,90 €" "1,20 €" "1,20 €" "1,15 €" ...
\end{verbatim}

Checking for missing values

\begin{Shaded}
\begin{Highlighting}[]
\FunctionTok{sum}\NormalTok{(}\FunctionTok{is.na}\NormalTok{(bakery))}
\end{Highlighting}
\end{Shaded}

\begin{verbatim}
## [1] 0
\end{verbatim}

\begin{Shaded}
\begin{Highlighting}[]
\CommentTok{\#No missing values}
\end{Highlighting}
\end{Shaded}

Checking for duplicated rows

\begin{Shaded}
\begin{Highlighting}[]
\FunctionTok{sum}\NormalTok{(}\FunctionTok{duplicated}\NormalTok{(bakery))}
\end{Highlighting}
\end{Shaded}

\begin{verbatim}
## [1] 0
\end{verbatim}

\begin{Shaded}
\begin{Highlighting}[]
\CommentTok{\#No duplicated rows}
\end{Highlighting}
\end{Shaded}

Cleaning up variable names

\begin{Shaded}
\begin{Highlighting}[]
\NormalTok{bakery }\OtherTok{\textless{}{-}} \FunctionTok{select}\NormalTok{(bakery, }\SpecialCharTok{{-}}\NormalTok{(}\StringTok{"X"}\NormalTok{)) }\SpecialCharTok{\%\textgreater{}\%} \CommentTok{\#Dropping unnecessary column}
  \FunctionTok{rename}\NormalTok{(}\AttributeTok{quantity =}\NormalTok{ Quantity) }\CommentTok{\#Rename column Quantity with a lowercase q to keep variable uniformity}
\FunctionTok{head}\NormalTok{(bakery)}
\end{Highlighting}
\end{Shaded}

\begin{verbatim}
##         date  time ticket_number              article quantity unit_price
## 1 2021-01-02 08:38        150040             BAGUETTE        1     0,90 €
## 2 2021-01-02 08:38        150040     PAIN AU CHOCOLAT        3     1,20 €
## 3 2021-01-02 09:14        150041     PAIN AU CHOCOLAT        2     1,20 €
## 4 2021-01-02 09:14        150041                 PAIN        1     1,15 €
## 5 2021-01-02 09:25        150042 TRADITIONAL BAGUETTE        5     1,20 €
## 6 2021-01-02 09:25        150043             BAGUETTE        2     0,90 €
\end{verbatim}

Convert \emph{unit\_price} to numeric for analysis and sale forecasting.
First getting rid of Euro symbol and replacing comma with a period

\begin{Shaded}
\begin{Highlighting}[]
\NormalTok{bakery}\SpecialCharTok{$}\NormalTok{unit\_price }\OtherTok{\textless{}{-}} \FunctionTok{gsub}\NormalTok{(}\StringTok{" €"}\NormalTok{, }\StringTok{" "}\NormalTok{, bakery}\SpecialCharTok{$}\NormalTok{unit\_price) }\CommentTok{\#Replaces Euro with empty space}
\NormalTok{bakery}\SpecialCharTok{$}\NormalTok{unit\_price }\OtherTok{\textless{}{-}} \FunctionTok{gsub}\NormalTok{(}\StringTok{","}\NormalTok{, }\StringTok{"."}\NormalTok{, bakery}\SpecialCharTok{$}\NormalTok{unit\_price) }\CommentTok{\#Replaces comma with period}
\FunctionTok{head}\NormalTok{(bakery)}
\end{Highlighting}
\end{Shaded}

\begin{verbatim}
##         date  time ticket_number              article quantity unit_price
## 1 2021-01-02 08:38        150040             BAGUETTE        1      0.90 
## 2 2021-01-02 08:38        150040     PAIN AU CHOCOLAT        3      1.20 
## 3 2021-01-02 09:14        150041     PAIN AU CHOCOLAT        2      1.20 
## 4 2021-01-02 09:14        150041                 PAIN        1      1.15 
## 5 2021-01-02 09:25        150042 TRADITIONAL BAGUETTE        5      1.20 
## 6 2021-01-02 09:25        150043             BAGUETTE        2      0.90
\end{verbatim}

Now we can convert \emph{unit\_price} to numeric

\begin{Shaded}
\begin{Highlighting}[]
\NormalTok{bakery}\SpecialCharTok{$}\NormalTok{unit\_price }\OtherTok{\textless{}{-}} \FunctionTok{as.numeric}\NormalTok{(bakery}\SpecialCharTok{$}\NormalTok{unit\_price)}
\FunctionTok{head}\NormalTok{(bakery)}
\end{Highlighting}
\end{Shaded}

\begin{verbatim}
##         date  time ticket_number              article quantity unit_price
## 1 2021-01-02 08:38        150040             BAGUETTE        1       0.90
## 2 2021-01-02 08:38        150040     PAIN AU CHOCOLAT        3       1.20
## 3 2021-01-02 09:14        150041     PAIN AU CHOCOLAT        2       1.20
## 4 2021-01-02 09:14        150041                 PAIN        1       1.15
## 5 2021-01-02 09:25        150042 TRADITIONAL BAGUETTE        5       1.20
## 6 2021-01-02 09:25        150043             BAGUETTE        2       0.90
\end{verbatim}

Calculating total of each purchase

\begin{Shaded}
\begin{Highlighting}[]
\NormalTok{bakery }\OtherTok{\textless{}{-}} \FunctionTok{mutate}\NormalTok{(bakery, }\AttributeTok{total =}\NormalTok{ unit\_price }\SpecialCharTok{*}\NormalTok{ quantity) }\CommentTok{\#Mutate creates a new column}
\FunctionTok{head}\NormalTok{(bakery)}
\end{Highlighting}
\end{Shaded}

\begin{verbatim}
##         date  time ticket_number              article quantity unit_price total
## 1 2021-01-02 08:38        150040             BAGUETTE        1       0.90  0.90
## 2 2021-01-02 08:38        150040     PAIN AU CHOCOLAT        3       1.20  3.60
## 3 2021-01-02 09:14        150041     PAIN AU CHOCOLAT        2       1.20  2.40
## 4 2021-01-02 09:14        150041                 PAIN        1       1.15  1.15
## 5 2021-01-02 09:25        150042 TRADITIONAL BAGUETTE        5       1.20  6.00
## 6 2021-01-02 09:25        150043             BAGUETTE        2       0.90  1.80
\end{verbatim}

For analysis, creating columns that separate date into year, month, day,
and day name

\begin{Shaded}
\begin{Highlighting}[]
\NormalTok{bakery}\SpecialCharTok{$}\NormalTok{date }\OtherTok{\textless{}{-}} \FunctionTok{as.Date}\NormalTok{(bakery}\SpecialCharTok{$}\NormalTok{date)}
\NormalTok{bakery }\OtherTok{\textless{}{-}} \FunctionTok{mutate}\NormalTok{(bakery, }
    \AttributeTok{year =} \FunctionTok{format}\NormalTok{(bakery}\SpecialCharTok{$}\NormalTok{date, }\StringTok{"\%Y"}\NormalTok{),}
    \AttributeTok{month =} \FunctionTok{format}\NormalTok{(bakery}\SpecialCharTok{$}\NormalTok{date, }\StringTok{"\%m"}\NormalTok{),}
    \AttributeTok{day =} \FunctionTok{format}\NormalTok{(bakery}\SpecialCharTok{$}\NormalTok{date, }\StringTok{"\%d"}\NormalTok{),}
    \AttributeTok{day\_name =} \FunctionTok{format}\NormalTok{(bakery}\SpecialCharTok{$}\NormalTok{date, }\StringTok{"\%A"}\NormalTok{))}
\FunctionTok{head}\NormalTok{(bakery)}
\end{Highlighting}
\end{Shaded}

\begin{verbatim}
##         date  time ticket_number              article quantity unit_price total
## 1 2021-01-02 08:38        150040             BAGUETTE        1       0.90  0.90
## 2 2021-01-02 08:38        150040     PAIN AU CHOCOLAT        3       1.20  3.60
## 3 2021-01-02 09:14        150041     PAIN AU CHOCOLAT        2       1.20  2.40
## 4 2021-01-02 09:14        150041                 PAIN        1       1.15  1.15
## 5 2021-01-02 09:25        150042 TRADITIONAL BAGUETTE        5       1.20  6.00
## 6 2021-01-02 09:25        150043             BAGUETTE        2       0.90  1.80
##   year month day day_name
## 1 2021    01  02 Saturday
## 2 2021    01  02 Saturday
## 3 2021    01  02 Saturday
## 4 2021    01  02 Saturday
## 5 2021    01  02 Saturday
## 6 2021    01  02 Saturday
\end{verbatim}

Since the dataset provides years 2021 and 2022, let's look at total in
both years

\begin{Shaded}
\begin{Highlighting}[]
\FunctionTok{ggplot}\NormalTok{(bakery, }\FunctionTok{aes}\NormalTok{(year, total, }\AttributeTok{fill =}\NormalTok{ year)) }\SpecialCharTok{+} \FunctionTok{geom\_bar}\NormalTok{(}\AttributeTok{stat =} \StringTok{"Identity"}\NormalTok{) }\SpecialCharTok{+} \FunctionTok{theme\_minimal}\NormalTok{() }\SpecialCharTok{+} \FunctionTok{scale\_fill\_brewer}\NormalTok{(}\AttributeTok{palette =} \StringTok{"Accent"}\NormalTok{) }\SpecialCharTok{+} \FunctionTok{labs}\NormalTok{(}\AttributeTok{title =} \StringTok{"Total Revenue (Year)"}\NormalTok{) }\SpecialCharTok{+} \FunctionTok{theme}\NormalTok{(}\AttributeTok{plot.title =} \FunctionTok{element\_text}\NormalTok{(}\AttributeTok{size =} \DecValTok{20}\NormalTok{, }\AttributeTok{face =} \StringTok{"bold"}\NormalTok{, }\AttributeTok{hjust =} \FloatTok{0.5}\NormalTok{))}
\end{Highlighting}
\end{Shaded}

\includegraphics{Bakery-Analysis_files/figure-latex/unnamed-chunk-12-1.pdf}

In months

\begin{Shaded}
\begin{Highlighting}[]
\FunctionTok{ggplot}\NormalTok{(bakery, }\FunctionTok{aes}\NormalTok{(month, total, }\AttributeTok{fill =}\NormalTok{ month)) }\SpecialCharTok{+} \FunctionTok{geom\_bar}\NormalTok{(}\AttributeTok{stat =} \StringTok{"Identity"}\NormalTok{) }\SpecialCharTok{+} \FunctionTok{theme\_minimal}\NormalTok{() }\SpecialCharTok{+} \FunctionTok{labs}\NormalTok{(}\AttributeTok{title =} \StringTok{"Total Revenue (Month)"}\NormalTok{) }\SpecialCharTok{+} \FunctionTok{theme}\NormalTok{(}\AttributeTok{plot.title =} \FunctionTok{element\_text}\NormalTok{(}\AttributeTok{size =} \DecValTok{20}\NormalTok{, }\AttributeTok{face =} \StringTok{"bold"}\NormalTok{, }\AttributeTok{hjust =} \FloatTok{0.5}\NormalTok{))}
\end{Highlighting}
\end{Shaded}

\includegraphics{Bakery-Analysis_files/figure-latex/unnamed-chunk-13-1.pdf}

In days

\begin{Shaded}
\begin{Highlighting}[]
\FunctionTok{ggplot}\NormalTok{(bakery, }\FunctionTok{aes}\NormalTok{(day\_name, total, }\AttributeTok{fill =}\NormalTok{ day\_name)) }\SpecialCharTok{+} \FunctionTok{geom\_bar}\NormalTok{(}\AttributeTok{stat =} \StringTok{"Identity"}\NormalTok{) }\SpecialCharTok{+} \FunctionTok{theme\_minimal}\NormalTok{() }\SpecialCharTok{+} \FunctionTok{labs}\NormalTok{(}\AttributeTok{title =} \StringTok{"Total Revenue (Day)"}\NormalTok{) }\SpecialCharTok{+} \FunctionTok{theme}\NormalTok{(}\AttributeTok{plot.title =} \FunctionTok{element\_text}\NormalTok{(}\AttributeTok{size =} \DecValTok{20}\NormalTok{, }\AttributeTok{face =} \StringTok{"bold"}\NormalTok{, }\AttributeTok{hjust =} \FloatTok{0.5}\NormalTok{))}
\end{Highlighting}
\end{Shaded}

\includegraphics{Bakery-Analysis_files/figure-latex/unnamed-chunk-14-1.pdf}

\begin{Shaded}
\begin{Highlighting}[]
\FunctionTok{ggplot}\NormalTok{(bakery, }\FunctionTok{aes}\NormalTok{(date, total)) }\SpecialCharTok{+} \FunctionTok{geom\_line}\NormalTok{() }\SpecialCharTok{+} \FunctionTok{theme\_minimal}\NormalTok{() }\SpecialCharTok{+} \FunctionTok{labs}\NormalTok{(}\AttributeTok{title =} \StringTok{"Total Revenue (Day)"}\NormalTok{) }\SpecialCharTok{+} \FunctionTok{theme}\NormalTok{(}\AttributeTok{plot.title =} \FunctionTok{element\_text}\NormalTok{(}\AttributeTok{size =} \DecValTok{20}\NormalTok{, }\AttributeTok{face =} \StringTok{"bold"}\NormalTok{, }\AttributeTok{hjust =} \FloatTok{0.5}\NormalTok{))}
\end{Highlighting}
\end{Shaded}

\includegraphics{Bakery-Analysis_files/figure-latex/unnamed-chunk-15-1.pdf}

\end{document}
